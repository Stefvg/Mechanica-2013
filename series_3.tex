\documentclass[10pt,a4paper]{article}
\usepackage[utf8]{inputenc}
\usepackage[english]{babel}
\usepackage{amsmath}
\usepackage{amsfonts}
\usepackage{amssymb}

\title{Oplossingen Mechanics 2013}
\author{TODO}


\begin{document}

\maketitle
\pagebreak
\tableofcontents
\pagebreak

\section{K2}
\subsection{Gegeven}
$m_{1} = 20kg$,  $m_{2} = 5 kg$,  $h = 0,3m$,  $v = 4m/s$,  $k=2000N/m $

\subsection{Gevraagd}
De indrukking van de veer $(\Delta l)$. 

\subsection{Oplossing}
We berekenen de kinetische energie van de kogel:\\
\[
E_{kin} = \frac{1}{2}*m*v^{2} = \frac{1}{2}*5kg*4^{2} = 40J
\]
De kogel heeft ook een potentiële energie op die bepaalde hoogte:\\
\[
E_{pot} = m*g*h = 5kg* 10 N/kg * 0,3m = 15J\\
\]
Nu moeten we nog berekenen hoe ver de veer is ingedrukt in het begin, hiervoor hebben we de potenti\'e le energie van de veer nodig:\\
\[
E_{veer} = \frac{1}{2}*k*\Delta l^{2}
\]
Nu we dit weten kunnen we de indrukking van de veer gaan berekenen aan de hand van deze 3 vergelijkingen:\\
\[
E_{veer} = E_{kin} + E_{pot}
\]
\[
\frac{1}{2}*2000N/m*\Delta l^{2} = 40J + 15J
\]
\[
\Delta l^{2} = \frac{55J}{1000N/m}
\]
\[
\Delta l = \sqrt{0,055}
\]
\[
\Delta l = 0,23m
\]

\end{document}