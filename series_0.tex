\documentclass[10pt,a4paper]{article}
\usepackage[utf8]{inputenc}
\usepackage[english]{babel}
\usepackage{amsmath}
\usepackage{amsfonts}
\usepackage{amssymb}

\title{Oplossingen Mechanics 2013}
\author{TODO}


\begin{document}

\maketitle
\pagebreak
\tableofcontents
\pagebreak


\section{Part 1}
\subsection{K1}

\subsection{K2}

\subsection{K3}
\subsubsection*{gegeven}
$a=0.5m$, $\theta=30^\circ$, $v_O=2\frac{m}{s}$.
\subsubsection*{gevraagd}
$v_{cx}$, $v_{cy}$, $a_{cx}$, $a_{cy}$
\subsubsection*{berekeningen}
De plaats van $C$ en $A$ in functie van de hoek valt af te leiden via Pythagoras.
\[
r_{c} =
\begin{pmatrix}
2a\cos{(\theta(t))}\\
2a\sin{(\theta(t))}
\end{pmatrix}
\text{ en }
r_{a} =
\begin{pmatrix}
2a\cos{(\theta(t))}\\
0
\end{pmatrix}
\]
Hieruit leiden we de snelheid en de versnelling af.
\[
v_{c} =
\begin{pmatrix}
-2a\sin{(\theta(t))}\cdot\omega(t)\\
2a\cos{(\theta(t))}\cdot\omega(t)
\end{pmatrix}
\text{ en }
r_{a} =
\begin{pmatrix}
-2a\sin{(\theta(t))}\cdot\omega(t)\\
0
\end{pmatrix}
\]
\[
a_{c} =
\begin{pmatrix}
\alpha(t)\cdot (-2a\sin({\theta(t)})-2a\cos({\theta(t))}\cdot\omega(t)^{2}\\
\alpha(t)\cdot (2a\cos5({\theta(t)})-2a\sin({\theta(t))}\cdot\omega(t)^{2}
\end{pmatrix}
\text{ en }
r_{a} =
\begin{pmatrix}
\alpha(t)\cdot (-2a\sin({\theta(t)})-2a\cos({\theta(t))}\cdot\omega(t)^{2}\\
0
\end{pmatrix}
\]
We zien dat de bewegingen van $A$ en $C$ precies gelijk zijn in de $x$ richting.
We kennen nu ook de volledige plaatsfunctie van $A$.
\[
r_{a} = 
\begin{pmatrix}
2a\cos{(\theta(t))}\\
0
\end{pmatrix}
=
\begin{pmatrix}
v_{O}\\
0
\end{pmatrix}
\cdot t
+
\begin{pmatrix}
2a\cos{\theta}\\
0
\end{pmatrix}
\]
Dit is in functie van t. Hier halen we $\theta$ in functie van $t$ uit.
\[
\theta(t) = \arccos{\left(\frac{v_{O}t + 2a\cos{\theta}}{2a}\right)}
\]
In deze vergelijking weten we alles.

\subsection{K4}

\subsection{B1}

\subsection{B2}
\subsubsection*{gegeven}
afstand $s$, $a_{max}$, $v_{max}$
\subsubsection*{gevraagd}
$t_{min}$
\subsubsection*{berekeningen}
De snelste manier om de afstand af te leggen, is door constant te versnellen aan de maximale versnelling tot aan de maximale snelheid, aan die snelheid blijven gaan, en dan op tijd beginnen vertragen aan de maximale vertraging.
\[
t_{tot_max_v} = \frac{v_{max}}{a_{max}}
\]
In die tijd is er
\[
\left(\frac{v}{a}\right)^{2}\cdot\frac{a}{2} = \frac{v^{2}}{2a}
\]
afgelegd. Om dit besluit te kunnen maken moeten we ervan uitgaan dat $s > \frac{v^{2}}{a}$
TODO, dont' get this yet.

\subsection{B3}

\section{Part 2}
\subsection{K1}
\subsubsection*{gegeven}
$|v|= 900\frac{km}{h} = 250\frac{m}{s}$
\subsubsection*{gevraagd}
$n: n \le r$
\subsubsection*{berekeningen}
We weten dat $a_n=\frac{v^2}{r}$ en  $a_t = \frac{dv}{dt}$.
\[
a_t=\frac{d(250)}{dt} = 0
\]
\[
a_n = \frac{|v|^2}{r} \le 4g
\]
\[
\frac{|v|^2}{4g}\le r
\]
Antwoord
\[
n= \frac{|v|^2}{4g} = \frac{250^2}{4 \cdot 10} = 1562.5m
\]

\subsection{K2}

\subsection{K3}
\subsubsection*{gegeven}
$r_{x}(t) = 3t^{2} m$, 
$v_{y}(t)= -\sqrt{13} \frac{m}{s}$
\subsubsection*{gevraagd}
$a_{t}$, $a_{n}$
\subsubsection*{berekeningen}
We leiden de plaats van het punt in de $x$ richting af naar de tijd en bekomen zo de snelheid en versnelling in de $x$ richting.
\[v_{x}(t)=6t\text{, }a_{x}(t) = 6\]
We leiden ook de versnelling van het punt in de $y$ richting af naar de tijd om de versnelling in de $y$ richting te vinden.
\[
a_{y} = 0
\]
Omdat de versnelling van het punt in de $y$ richting nul is, weten we dat $a = a_{x}$
\\We weten dat
\[
a_{t} = \frac{d|\vec{v}|}{dt} = \frac{d\sqrt{(6t)^{2} + (\sqrt{13})^{2}}}{dt} = \frac{36t}{\sqrt{36t^{2}+13}}
\]
en ook dat
\[
a_{n} = \sqrt{a^{2} + a_{t}^{2}} = \sqrt{a^{2} + \left(\frac{36t}{\sqrt{36t^{2}+13}}\right)^{2}}
\]
In beide van deze vergelijkingen kennen we alle 'onbekenden'. Na invullen:
\[
a_{n} = 3.1\frac{m}{s^{2}} \text{ en } a_{t} = 5.14\frac{m}{s^{2}}
\]

\subsection{B1}
\subsubsection*{gegeven}
$R=-0.08m$, $v_{py} = -0.06 \frac{m}{s}$, $\theta = 60^\circ$
\subsubsection*{gevraagd}
$\omega(\theta)$, $\alpha(\theta)$, $a_{p}$
\subsubsection*{berekeningen}
We kunnen de plaatsfunctie van $P$ bepalen.
\[
r_{p}(t)=
\begin{pmatrix}
-\cos{(\theta(t)}\cdot R\\
-\sin{(\theta(t)}\cdot R
\end{pmatrix}
\] 
We kunnen dit afleiden naar de tijd, maar we weten dat de snelheid van $P$ in de $y$ richting constant is:
\[
v_{p}(t)=
\begin{pmatrix}
\omega(t)\sin{(\theta(t))}\cdot R\\
v_{py}
\end{pmatrix}
\]
met $v_{py} = -\omega(t)\cos{(\theta(t))}\cdot R$.
\[
a_{p}(t)=
\begin{pmatrix}
R\cdot(\cos{(\theta(t))}\omega(t)^{2} + \alpha(t)\sin{(\theta(t))})\\
R\cdot(-\sin{(\theta(t))}\omega(t)^{2} + \alpha(t)\cos{(\theta(t))})
\end{pmatrix}
\]
met $a_{py} = 0$ want $v_{py}$ is constant.
\\\\
Hieruit kunnen we halen dat 
\[
\omega(t) = -\frac{v_{py}}{\cos{(\theta(t))}\cdot R}
\]
Dit is $1.5 \frac{rad}{s}$ in de gegeven positie.\\
Nu weten we dat
\[
0=R\cdot(-\sin{(\theta(t))}\omega(t)^{2} + \alpha(t)\cos{(\theta(t))})
\]
en dat wordt:
\[
\alpha(t) = -\tan{(\theta(t))}\omega(t)^{2}
\]
Hierin kennen we alle 'onbekenden'. Na invullen is $\alpha = -3.9\frac{rad}{s^{2}}$.\\
We weten dat
\[
a_{t} = R\cdot\alpha(t) \text{ en } a_{n} = R\cdot \omega(t)^{2}
\]
en
\[
a = \sqrt{a_{t}^{2} + a_{n}^{2}}
\]
Hierin kennen we opnieuw alle 'onbekenden'. Na invullen blijkt dat $a=0.36\frac{m}{s^{2}}$ op het gegeven moment.

\subsection{B2}

\subsection{B3}

\section{Part 3}
\subsection{K1}

\subsection{K2}

\subsection{K3}

\subsection{K4}
\subsubsection*{gegeven}
$|OA|=0.35m$, $|AB| = 0.60m$, $\theta = \hat{AOB} = 45^\circ$, $180^\circ-\gamma = \hat{BCO} = 180^\circ-60^\circ$, $\omega_{OA} = 6\frac{rad}{s}$
\subsubsection*{gevraagd}
$\omega_{CB}$
\subsubsection*{berekeningen}
We zullen $\omega_{BC}$ bepalen met een tussenstap.
Stap 1: We bepalen een punt waarvan we de snelheid zoeken. We kiezen hier voor $P=A$.\\
Stap 2: We bepalen het bewegende assenstelsel $O'x’y’z’$ We kiezen hier voor een translerend bewegend assenstelsel met $O'=A$ en $x'=AB$. Omdat dit een translerend assenstelsel is, dat niet roteerd, is: $e_{x} = e_{x'}$, $e_{z} = e_{z'}$, $e_{z} = e_{z'}$.
We weten dat:
\[
\vec{v_{ba}} = \vec{v_{bs}} + \vec{v_{br}}
\]
Nu zien we dat:
\[
\vec{v_{bs}} = \vec{v_{A}} = \omega_{OA} \times (\vec{r_{A}} - \vec{r_{O}}) = 
\begin{vmatrix}
\vec{e_{x}} & \vec{e_{x}} & \vec{e_{x}}\\
0 & 0 & \omega_{OA}\\
|OA|\cos{\theta(t)} & |OA|\sin{\theta(t)} & 0
\end{vmatrix} 
=
\begin{pmatrix}
-\omega_{OA} \cdot |OA|\sin{\theta(t)}\\
-\omega_{OA} \cdot |OA|\cos{\theta(t)}
0
\end{pmatrix}
\]
We weten ook dat:
\[
\vec{v_{br}} = \omega_{AB} \times (\vec{r_{B}} - \vec{r_{A}}) = 
\begin{vmatrix}
\vec{e_{x}} & \vec{e_{x}} & \vec{e_{x}}\\
0 & 0 & \omega_{AB}\\
|AB| & 0 & 0
\end{vmatrix} 
=
\begin{pmatrix}
0\\
-\omega_{AB} \cdot |AB|\\
0
\end{pmatrix}
\]
Ten slotte weten we nog dat:
\[
\vec{v_{ba}} = \omega_{CB} \times (\vec{r_{B}} - \vec{r_{C}}) = 
\begin{vmatrix}
\vec{e_{x}} & \vec{e_{x}} & \vec{e_{x}}\\
0 & 0 & \omega_{CB}\\
|BC|\cos{\gamma(t)} & |BC|\sin{\gamma(t)} & 0
\end{vmatrix} 
=
\begin{pmatrix}
-\omega_{CB} \cdot |BC|\sin{\gamma(t)}\\
-\omega_{CB} \cdot |BC|\cos{\gamma(t)}\\
0
\end{pmatrix}
\]
Als we dit allemaal samen stellen is het:
\[
\left\lbrace
\begin{array}{l r}
-\omega_{OA} \cdot |OA|\sin{\theta(t)} + 0 &= -\omega_{CB} \cdot |BC|\sin{\gamma(t)}\\
-\omega_{OA} \cdot |OA|\cos{\theta(t)} + -\omega_{AB} \cdot |AB| &= -\omega_{CB} \cdot |BC|\cos{\gamma(t)}\\
\end{array}
\right.
\]
De eerste vergelijking hier in:
\[
-\omega_{OA} \cdot |OA|\sin{\theta(t)}= -\omega_{CB} \cdot |BC|\sin{\gamma(t)}
\]
is om te vormen naar
\[
\omega_{CB} = \frac{\omega_{OA} \cdot |OA|\sin{\theta(t)}}{|BC|\sin{\gamma(t)}}
\]
In deze vergelijking kennen we enkel $|BC|$ niet.\\
We kunnen $|OC|$ berekenen via de cosinusregel: 
\[
|OB| = \sqrt{|OA|^{2} + |AB|^{2} - 2 \cdot |OA| \cdot |AB| \cdot \cos(\hat{OAB})}
\]
Nu weten we dat $|OB| = 0.883m$
We weten volgens de sinusregel dat:
\[
\frac{\sin(45^\circ-\hat{BOC})}{|AB|} = \frac{\sin(\hat{OAB})}{|OB|} \Leftrightarrow \hat{BOC} = 16.3^\circ
\]
Volgens diezelfde sinusregel weten we nu ook dat
\[
\frac{\sin(\hat{BOC})}{|OB|} = \frac{\sin(\hat{BOC})}{|BC|}
\Leftrightarrow |BC| = 0.28m
\]
Nu we ook $|BC|$ kennen rest er ons enkel nog de formule voor $\omega_{CB}$ in te vullen.
\[
\omega_{CB} = \frac{\omega_{OA} \cdot |OA|\sin{\theta(t)}}{|BC|\sin{\gamma(t)}} = 6\;\frac{rad}{s}
\]

\subsection{B1}
\subsubsection*{gegeven}
$d(A,B) = d = 200m$\\
$v_{zwemmer}=1.8\frac{km}{u}=0.5\frac{m}{s}$\\
Geval a,b:\\
$v_{water}= 0.54\frac{km}{u}=0.15\frac{m}{s}$\\
Geval c:\\
$v_{water}= 0\frac{m}{s}$\\\\
Geval a: \[\widehat{v_{water},AB} = 0^\circ\]
Geval b: \[\widehat{v_{water},AB} = 30^\circ\]

\subsubsection*{gevraagd}
$\Delta t_{A\rightarrow B} + \Delta t_{B\rightarrow A}$

\subsubsection*{berekeningen}
\[
t = \frac{\Delta x}{\Delta t}
\]
a)\\
\[
\Delta t_{A\rightarrow B} = \frac{d}{v_{zwemmer} + \cdot v_{water}} = \frac{200}{0.5+0.15} = 307 s
\]
\[
\Delta t_{B\rightarrow A} = \frac{d}{v_{zwemmer} + \cdot v_{water}} = \frac{200}{0.5-0.15} = 571 s
\]
Antwoord:
$\Delta t_{A\rightarrow B} + \Delta t_{B\rightarrow A} = 879$ s.\\\\
b)\\
De zwemmer moet nu de hoek waaronder hij zwemt ten opzichte van $AB$ zodat hij in een rechte lijn zwemt voor een waarnemer op de oever.
\[
\sin{\theta}\cdot v_{zwemmer} + \sin{(-30^\circ)}\cdot v_{water} = 0 \Leftrightarrow \theta=\arcsin{\frac{(\sin{(30^\circ)}\cdot v_{water})}{v_{zwemmer}}}
\]
\[
\theta = 8.63^\circ
\]
\[
\Delta t_{A\rightarrow B} = \frac{d}{\cos{\theta}v_{zwemmer} + \cos{(-30^\circ)}v_{water}}=320 s
\]
\[
\Delta t_{B\rightarrow A} = \frac{d}{\cos{\theta}v_{zwemmer} - \cos{(-30^\circ)}v_{water}}=549 s
\]
Antwoord:
$\Delta t_{A\rightarrow B} + \Delta t_{B\rightarrow A} = 869$ s.\\
c)\\
\[
\Delta t_{A\rightarrow B} = \frac{d}{v_{zwemmer}} = \Delta t_{B\rightarrow A} = 400
\]
Antwoord:
$\Delta t_{A\rightarrow B} + \Delta t_{B\rightarrow A} = 800$ s.\\

\subsection{B2}

\subsection{B3}

\subsection{B4}


\end{document}