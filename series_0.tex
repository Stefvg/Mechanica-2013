\documentclass[10pt,a4paper]{article}
\usepackage[utf8]{inputenc}
\usepackage[english]{babel}
\usepackage{amsmath}
\usepackage{amsfonts}
\usepackage{amssymb}

\title{Oplossingen Mechanics 2013}
\author{TODO}


\begin{document}

\maketitle
\pagebreak
\tableofcontents
\pagebreak


\section{Part 1}
\subsection{K1}

\subsection{K2}

\subsection{K3}
\subsubsection*{gegeven}
$a=0.5m$, $\theta=30^\circ$, $v_O=2\frac{m}{s}$.
\subsubsection*{gevraagd}
$v_{cx}$, $v_{cy}$, $a_{cx}$, $a_{cy}$
\subsubsection*{berekeningen}
De plaats van $C$ en $A$ in functie van de hoek valt af te leiden via Pythagoras.
\[
r_{c} =
\begin{pmatrix}
2a\cos{(\theta(t))}\\
2a\sin{(\theta(t))}
\end{pmatrix}
\text{ en }
r_{a} =
\begin{pmatrix}
2a\cos{(\theta(t))}\\
0
\end{pmatrix}
\]
Hieruit leiden we de snelheid en de versnelling af.
\[
v_{c} =
\begin{pmatrix}
-2a\sin{(\theta(t))}\cdot\omega(t)\\
2a\cos{(\theta(t))}\cdot\omega(t)
\end{pmatrix}
\text{ en }
r_{a} =
\begin{pmatrix}
-2a\sin{(\theta(t))}\cdot\omega(t)\\
0
\end{pmatrix}
\]
\[
a_{c} =
\begin{pmatrix}
\alpha(t)\cdot (-2a\sin({\theta(t)})-2a\cos({\theta(t))}\cdot\omega(t)^{2}\\
\alpha(t)\cdot (2a\cos5({\theta(t)})-2a\sin({\theta(t))}\cdot\omega(t)^{2}
\end{pmatrix}
\text{ en }
r_{a} =
\begin{pmatrix}
\alpha(t)\cdot (-2a\sin({\theta(t)})-2a\cos({\theta(t))}\cdot\omega(t)^{2}\\
0
\end{pmatrix}
\]
We zien dat de bewegingen van $A$ en $C$ precies gelijk zijn in de $x$ richting.
We kennen nu ook de volledige plaatsfunctie van $A$.
\[
r_{a} = 
\begin{pmatrix}
2a\cos{(\theta(t))}\\
0
\end{pmatrix}
=
\begin{pmatrix}
v_{O}\\
0
\end{pmatrix}
\cdot t
+
\begin{pmatrix}
2a\cos{\theta}\\
0
\end{pmatrix}
\]
Dit is in functie van t. Hier halen we $\theta$ in functie van $t$ uit.
\[
\theta(t) = \arccos{\left(\frac{v_{O}t + 2a\cos{\theta}}{2a}\right)}
\]
In deze vergelijking weten we alles.

\subsection{K4}

\subsection{B1}

\subsection{B2}

\subsection{B3}

\section{Part 2}
\subsection{K1}
\subsubsection*{gegeven}
$|v|= 900\frac{km}{h} = 250\frac{m}{s}$
\subsubsection*{gevraagd}
$n: n \le r$
\subsubsection*{berekeningen}
We weten dat $a_n=\frac{v^2}{r}$ en  $a_t = \frac{dv}{dt}$.
\[
a_t=\frac{d(250)}{dt} = 0
\]
\[
a_n = \frac{|v|^2}{r} \le 4g
\]
\[
\frac{|v|^2}{4g}\le r
\]
Antwoord
\[
n= \frac{|v|^2}{4g} = \frac{250^2}{4 \cdot 10} = 1562.5m
\]

\subsection{K2}

\subsection{K3}

\subsection{B1}

\subsection{B2}

\subsection{B3}

\section{Part 3}
\subsection{K1}

\subsection{K2}

\subsection{K3}

\subsection{K4}

\subsection{B1}
\subsubsection*{gegeven}
$d(A,B) = d = 200m$\\
$v_{zwemmer}=1.8\frac{km}{u}=0.5\frac{m}{s}$\\
Geval a,b:\\
$v_{water}= 0.54\frac{km}{u}=0.15\frac{m}{s}$\\
Geval c:\\
$v_{water}= 0\frac{m}{s}$\\\\
Geval a: \[\widehat{v_{water},AB} = 0^\circ\]
Geval b: \[\widehat{v_{water},AB} = 30^\circ\]

\subsubsection*{gevraagd}
$\Delta t_{A\rightarrow B} + \Delta t_{B\rightarrow A}$

\subsubsection*{berekeningen}
\[
t = \frac{\Delta x}{\Delta t}
\]
a)\\
\[
\Delta t_{A\rightarrow B} = \frac{d}{v_{zwemmer} + \cdot v_{water}} = \frac{200}{0.5+0.15} = 307 s
\]
\[
\Delta t_{B\rightarrow A} = \frac{d}{v_{zwemmer} + \cdot v_{water}} = \frac{200}{0.5-0.15} = 571 s
\]
Antwoord:
$\Delta t_{A\rightarrow B} + \Delta t_{B\rightarrow A} = 879$ s.\\\\
b)\\
De zwemmer moet nu de hoek waaronder hij zwemt ten opzichte van $AB$ zodat hij in een rechte lijn zwemt voor een waarnemer op de oever.
\[
\sin{\theta}\cdot v_{zwemmer} + \sin{(-30^\circ)}\cdot v_{water} = 0 \Leftrightarrow \theta=\arcsin{\frac{(\sin{(30^\circ)}\cdot v_{water})}{v_{zwemmer}}}
\]
\[
\theta = 8.63^\circ
\]
\[
\Delta t_{A\rightarrow B} = \frac{d}{\cos{\theta}v_{zwemmer} + \cos{(-30^\circ)}v_{water}}=320 s
\]
\[
\Delta t_{B\rightarrow A} = \frac{d}{\cos{\theta}v_{zwemmer} - \cos{(-30^\circ)}v_{water}}=549 s
\]
Antwoord:
$\Delta t_{A\rightarrow B} + \Delta t_{B\rightarrow A} = 869$ s.\\
c)\\
\[
\Delta t_{A\rightarrow B} = \frac{d}{v_{zwemmer}} = \Delta t_{B\rightarrow A} = 400
\]
Antwoord:
$\Delta t_{A\rightarrow B} + \Delta t_{B\rightarrow A} = 800$ s.\\

\subsection{B2}

\subsection{B3}

\subsection{B4}


\end{document}