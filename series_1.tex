\documentclass[10pt,a4paper]{article}
\usepackage[utf8]{inputenc}
\usepackage[english]{babel}
\usepackage{amsmath}
\usepackage{amsfonts}
\usepackage{amssymb}

\title{Oplossingen Mechanics 2013}
\author{TODO}


\begin{document}

\maketitle
\pagebreak
\tableofcontents
\pagebreak

\section*{K2}
\subsection*{Gegeven}
$m_{man} = 70kg$, $m_{platform} = 30kg$, $a=2\frac{m}{s^2}$ 
\subsection*{Gevraagd}
$F_{trek}$, $F_n$
\subsection*{Oplossing}
Maak de man en het platform apart vrij\\
Man:
\[
F_{trek} + F_n - G_{man} = m_{man} a
\]
Platform
\[
F_{trek} - G_{platform} - F_n = m_{platform}a
\]
met
\[
G_{man} = m_{man}g \text{ en } G_{platform} = m_{platform}g
\]
Als we dit stelsel oplossen naar $F_n$ en $F_trek$ krijgen we
\[
F_{trek} = 550N \text{ en } F_n = 220N
\]

\section*{K3}
\subsection*{Gegeven}
F=100N, m1= 1kg, m2= 9kg, l0=1m k =20 n/m
\subsection*{Gevraagd}
l
\subsection*{Oplossing}
De kracht op het geheel:\\
$\vec{F} = m_t * \vec{a}\text{\hspace{2cm} met $m_t$ het totaal gewicht van het systeem}\\
\vec{a} = \frac{100}{9+1} = 10 m/s^2 \vec{e}_x$\\

\noindent De krachten in $m_1$ zijn dan:\\
$\rightarrow$ Normaal kracht + Zwaartekracht, maar deze heffen elkaar op\\
$\rightarrow$ De veerkracht $\vec{F}_v$\\

\noindent De veerkracht is de enige kracht en zorgt dus integraal voor de versnelling:\\
$F_v = m_1*a\\
F_v = 1*10 = 10 N$\\

\noindent Omdat $\vec{F}_v$ constant is kan de lengte van de veer gehaald worden uit:\\
$F_v = k * (l_0 - l)\\
10 = 20 *(1 -l)\\
\Rightarrow l =0.5m$
\section*{B4}
We kunnen het tweede postulaat van Newton op beide blokken toepassen, in beide dimensies:\\
Blok $m_1$:
\[
\begin{array}{l l}
x: & F- F_{w_{grond,m_1}} -F_{w_{m_1,m_2}}=m_1a\\
y: & F_{n_{grond,m_1}}-F_{n_{m_1,m_2}}-F_{z_{m_1}}=0\\
\end{array}
\]
Blok $m_2$:
\[
\begin{array}{l l}
x: & F_{w_{m_1,m_2}} - \cos(\theta)F_{trek} = 0\\
y: & F_{n_{m_1,m_2}} + \sin(\theta)F_{trek} - F_{z_{m_2}}=0\\
\end{array}
\]
Dit lijkt een monster van een stelsel - (Dat is het ook!) - maar er zijn een aantal dingen die we weten.
\[
F_{z_{m_1}} = m_1g \text{ en } F_{z_{m_2}} = m_1g
\]
\[
F_{w_{m_1,m_2}} = f_2F_{n_{m_1,m_2}} \text{ en } F_{w_{grond,m_1}} = f_1F_{n_{grond,m_1}}
\]

Als we dit allemaal invullen krijgen we:
\[
\left\lbrace
\begin{array}{l l l l l}

f_2F_{n_{m_1,m_2}} &- \cos(\theta)F_{trek} &+ 0 &+ 0 &= 0\\
F_{n_{m_1,m_2}} &+ \sin(\theta)F_{trek} &+ 0 &+ 0 &= m_2g\\
 f_2F_{n_{m_1,m_2}} &+ 0 &+ f_1F_{n_{grond,m_1}} &+ m_1a &= F\\
-F_{n_{m_1,m_2}} &+ 0 &+F_{n_{grond,m_1}} &+ 0 &= m_1g\\
\end{array}
\right.
\]
Dit kunnen we omvormen tot een stelsel dat makkelijk op te lossen is.

\[
\left(
\begin{array}{l l l l | l}
f_2 & -\cos(\theta) & 0 & 0 & 0\\
1 &  \sin(\theta) & 0 & 0 & m_2g\\
f_2 & 0 & f_1 & m_1 & F\\
-1 & 0 & 1 & 0 & m_1g
\end{array}
\right)
\]
Als we dit uitrekenen zien we dat $a = 34.9$ en $F_{trek} = 8.3$. Dit lijkt fout.

\end{document}