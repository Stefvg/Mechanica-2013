\documentclass[10pt,a4paper]{article}
\usepackage[utf8]{inputenc}
\usepackage[english]{babel}
\usepackage{amsmath}
\usepackage{amsfonts}
\usepackage{amssymb}

\title{Oplossingen Mechanics 2013}
\author{TODO}


\begin{document}

\maketitle
\pagebreak
\tableofcontents
\pagebreak

\section*{K3}
\section*{B4}
We kunnen het tweede postulaat van Newton op beide blokken toepassen, in beide dimensies:\\
Blok $m_1$:
\[
\begin{array}{l l}
x: & F - F_{w_{m_1,m_2}} - F_{w_{m_1,grond}} = m a_{m_1x}\\
y: & F_{n_{m_1}} -F_{m_1,m_2} - F_{z_{m_1}} = m a_{m_1y}\\
\end{array}
\]
Blok $m_2$:
\[
\begin{array}{l l}
x: & F_{w_{m_1,m_2}} - \cos(\theta)F_{trek} = m a_{m_2x}\\
y: & \sin(\theta)F_{trek} + F_{m_1,m_2} - F_{z_{m_2}} = m a_{m_2y}\\
\end{array}
\]
Dit lijkt een monster van een stelsel - (Dat is het ook!) - maar er zijn een aantal dingen die we weten.
\[
a_{m_1y} = 0 \text{ ,  } m a_{m_2x}=0 \text{ ,  }  m a_{m_2y}=0
\]
\[
F_{w_{m_1,m_2}} = f_2 F_{m_1,m_2} \text{ ,  }F_{w_{m_1,grond}} = f_1 F_{n_{m_1}}
\]
\[
F_{z_{m_1}} = m_1 g \text{ ,  } F_{z_{m_2}} = m_2 g
\]

Als we dit allemaal invullen krijgen we:
\[
\left\lbrace
\begin{array}{l}
F - f_2 F_{m_1,m_2} - f_1 F_{n_{m_1}} = m a_{m_1x}\\
F_{n_{m_1}} - F_{m_1,m_2} -  m_1 g  = 0\\
f_2 F_{m_1,m_2} - \cos(\theta)F_{trek} = 0\\
\sin(\theta)F_{trek} + F_{m_1,m_2} - m_2 g = 0\\
\end{array}
\right.
\]
Dit kunnen we omvormen tot een stelsel dat makkelijk op te lossen is.

\[
\left\lbrace
\begin{array}{l l l l l}
f_2 F_{m_1,m_2} &+ f_1 F_{n_{m_1}} &+ 0 & m a_{m_1x} &=F\\
- F_{m_1,m_2} &+ F_{n_{m_1}}  &+ 0 &+ 0 &= m_1 g\\
f_2 F_{m_1,m_2} &+ 0 &- \cos(\theta)F_{trek} &+ 0 &= 0\\
F_{m_1,m_2} &+ 0 &+\sin(\theta)F_{trek} &+ 0 &= m_2 g\\
\end{array}
\right.
\]
Als we dit uitrekenen zien we dat het fout is. FTS.

\end{document}